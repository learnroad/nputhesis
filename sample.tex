%% UTF-8
\documentclass[twoside, workbib, UTF8, phd, workbib=biblatex]{nputhesis}

\usepackage{lipsum}

% 该宏包已包含在 TeXLive 和 MikTeX 中.
% gbnamefmt 参数需要 v1.0k 版本.
% \usepackage[backend=biber,
%             style=gb7714-2015,
% %           gbnamefmt=none,
%             doi=false]{biblatex}

\addbibresource{ref.bib}        % biblatex 命令. 添加参考文献文件, 必须加后缀.

%%%%%%%%%%%%%%%%%%%%%%%%%%        workbib      %%%%%%%%%%%%%%%%%%%%%%%%%%%%%%
% workbib 选项
% + none
%    不自动生成成果
% + biblatex 
%    使用 biblatex 处理参考文献并可以自动生成成果. 如需自动生成成果可以
%    在 \Work 章节可以使用命令 \npuprintpub{<name-list>}{<bib-key-list>}.
%    上述参数 <name-list> 用于名字加黑
%    e.g.
%        \npuprintpub{ZhangSan, 张三}{A2016, B2017, C2018, D2019}
% + multibib
%    使用 multibib 自动生成成果
%    在 \Work 章节, 使用 \npuprintpub[<bibstyle>]{<bib-file>}{<bib-key-list>}
%    e.g.
%        \npuprintpub[nputhesis]{ref}{A2016, B2017, C2018, D2019}

% 注1: ------------------自动生成并不符合格式规定--------------------------
%%%%%%%%%%%%%%%%%%%%%%%%%%        workbib      %%%%%%%%%%%%%%%%%%%%%%%%%%%%%%

\newtheorem{thm}{定理}[section]
\theoremstyle{plain}
\newtheorem{plain-thm}{定理(plain)}[section]

\schoolno{10699}
\classno{O.242}
\secretlevel{公开}
\authorno{2018999999}

\title[\LaTeX\ Template for Thesis of NPU]{西北工业大学硕博士论文\LaTeX 模板}

\author[San Zhang]{张\,\,三}
\major[Mathematics]{数学}
\supervisor[Si Li]{李四}
\applydate[April 2018]{2018~年~4~月}
\support{本文研究得到某某基金(编号:XXXXXXX)资助。}

\begin{document}
\makecover    % 生成中英文封面.
\frontmatter  % 分割封面与前言部分.

% 中文摘要
\begin{abstract}
  \LaTeX 是一种基于 \TeX 的排版系统, 它非常适用于生成高印刷质量的科技和数学,
  化学类文档. \LaTeX 把排版的细节隐藏在若干样式之后, 以内容的逻辑结构统帅复杂
  的格式, 现已成为科技文写作特别是数学写作的重要工具之一. NPU 仅提供了博士论
  文的 Word 模板, 所以我们根据官方 Word 模板制作该模板, 以满足使用 \LaTeX 写作
  论文的需求.

  本模板基本实现了官方格式要求: 封皮, 页眉页脚, 章节标题格式, 参考文献格式等.

  { % 这怎么可能写两页? 只能 lipsum 了.
    \noindent\hrulefill\\
    {\centerline {\it lipsum begin}}
    \lipsum[1-5]
    {\centerline{\it lipsum end}}
    \noindent\hrulefill
  }
  \begin{keywords}
    论文模板, \LaTeX, 西工大
  \end{keywords}
\end{abstract}

% 英文摘要
\begin{Abstract}
  { % some meaningless words.
    \noindent\hrulefill\\
    {\centerline {\it lipsum begin}}
    \lipsum[1-4]
    {\centerline{\it lipsum end}}
    \noindent\hrulefill
  }
  \begin{Keywords}
    Thesis Template, \LaTeX, NPU
  \end{Keywords}
\end{Abstract}

\tableofcontents    % 目录
\printnomenclature  % 符号命名表 添加符号可用 \nomenclature{<sym>}{<text explanation>}
\mainmatter         % 分割目录和正文部分

% 添加符号  这些可在正文中任意的地方
\nomenclature{$\mathbb{R}^n$}{$n$ 维欧几里得空间}
\nomenclature{$\varOmega$}{空间 $\mathbb{R}^n$ 中的开集}
\nomenclature{$L^p(\varOmega$}{Lebesgue 空间}
\nomenclature{$W^{m,p}(\varOmega)$}{Sobolev 空间}
\nomenclature{$H^m(\varOmega)$}{具有 $m$ 阶 $L^2$ 导数 的 Sobolev 空间}
\nomenclature{$C^k(\varOmega)$}{具有 $k$ 阶连续导数的函数空间}
\nomenclature{$C^k_0(\varOmega)$}{具有 $k$ 阶连续导数且有紧支集的函数空间}

\chapter{nputhesis 简介}

\section{\TeX 和 \LaTeX 介绍}
关于 \TeX 和 \LaTeX 请参考 \cite{Knuth1986,Lamport1994,Liu2013}, 其中 \cite{Liu2013} 最适合入门.

\section{nputhesis 依赖}
如下表格给出了测试编译通过的环境
\begin{table}[h]
  \caption{测试环境\cite{Liu2013}}
  \centering
  \begin{tabular}{cccc}
    \toprule
    操作系统    & \TeX 系统   & 版本  & 引擎\\
    \midrule
    Windows 10  & TeXLive     & 2017  & xelatex\\
    \bottomrule
  \end{tabular}
\end{table}

% \nomenclature{$H^{1}_{0}(\Omega)$}{Hilbert 空间}  % 向符号表添加内容
\section{插入图片}
\lipsum[1-3]
\begin{figure}
    \centering
    \includegraphics[width=0.3\textwidth]{figures/fig1-1.pdf}
    \includegraphics[width=0.3\textwidth]{figures/fig1-1.pdf}
    \caption{样例图}
\end{figure}
\begin{figure}
    \centering
    \includegraphics[width=0.5\textwidth]{figures/fig2.pdf}
    \caption{另一个图}
\end{figure}
\lipsum[4-6]
\section{定理环境}
\lipsum[1]
\begin{thm}
    这里是定理环境使用 `nputheorem' 格式.
\end{thm}
\lipsum[2]
\begin{plain-thm}
    这里是 `amsthm' 默认的定理环境, 使用 `plain' 格式.
\end{plain-thm}
\lipsum[3]

\chapter{实现}
\section{思路}
\section{代码}
\nocite{A2016, B2017, C2018, D2019}
\backmatter
% 根据实际宏包使用情况自动选择, 不需更改
\npuchoosebib{%
  \printbibliography             % 如果使用了 biblatex 宏包
}{%
  \bibliographystyle{nputhesis}  % 传统输出参考文献方式
  \bibliography{ref}             % 传统输出参考文献方式
}
\Appendix  % 如果有多个附录, 可重复使用该命令, 自动按字母编号.

\Thanks     % 致谢

\Work
% workbib 的三个选项

% biblatex 方式 \npuprintpub{<bib-key-list>}
\npuprintpub{ZhangSan, 张三}{A2016, B2017, C2018, D2019}

% multibib 方式 \npuprintpub[<bibstyle>]{<bib-file>}{<bib-key-list>}
% \npuprintpub[nputhesis]{ref}{A2016, B2017}

% none  方式 即手动书写方式
% \papersection  % 以下填写发表论文情况
% 1. 论文1

% 2. 论文2

\researchsection % 以下填写参加科研情况
1. 我的研究项目1

2. 我的研究项目2

\statement
\end{document}
